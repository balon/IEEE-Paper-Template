\section{Examples}\label{sec:examples}

% we can prepare tables in other files to avoid clutter
\textbf{Single-Column table:}
\begin{table}[h]
\begin{center}
    \begin{tabular}{ | c | c | c | c |}
    \hline
    \textbf{Col1} & \textbf{Column 2} & \textbf{Column 3} & \textbf{Column 4}\\ \specialrule{.1em}{.05em}{.05em} 
    
    % Row Entry #1
    [?]
    &
    & 
    & 
    \\ \hline 
    
    \end{tabular}
\end{center}
\caption{Small Table caption}
\label{table:smalltABLE}
\end{table} \hfill \break 

\textbf{Multi-Column table on next page.}
% Generate a table that will be inserted at the top of the next-page 
% will float, and fill 2 columns 
\begin{table*}
\begin{center}
    \begin{tabular}{ | c | p{6cm} | p{5cm} | p{5cm} |}
    \hline
    \textbf{Col1} & \textbf{Column 2} & \textbf{Column 3} & \textbf{Column 4}\\ \specialrule{.1em}{.05em}{.05em} 
    
    & \textit{\textbf{Section Header}} & \textit{-}& \textit{-} \\ \hline
    [?]
    &
    & 
    & 
    \\ \hline 
    
    \end{tabular}
\end{center}
\caption{Large table caption}
\label{table:largeTable}
\end{table*} \hfill \break 

\textbf{Command Examples:} \hfill \break 
\todo{To complete.} \hfill \break 
\update{To change.} \hfill \break 
\rewrite{To refactor.} \hfill \break 
\thoughts{Details you can provide.} \hfill \break 
Reminder to cite \needcite. \hfill \break 

\textbf{Single-Column figure:} \hfill \break 
\begin{figure}[h]
  \centering
  \includegraphics[scale=0.2]{_graphics/smExample.png}
  \caption{Small Photo Caption\footnotemark}
  \label{fig:smPhoto}
\end{figure}

% example of inserting a footnote based on where you drop a mark. Floating images in multi-column papers must be referenced this way.
\footnotetext{Image Info. (url: \url{https://example.com/image.png}) \urlDate}


\textbf{Multi-Column figure on next page.} \hfill \break 
\begin{figure*}[h]
  \centering
  \includegraphics[width=\textwidth]{_graphics/lgExample.png}
  \caption{Large Photo Caption}
  \label{fig:lgPhoto}
\end{figure*}

We can write down here, below the figure going onto the next page. When an image or table floats, it will fill the topmost column of the next page. We can demonstrate further using lipsum (dummy text). \hfill \break
\lipsum[1-2] \hfill \break \break 

%finally, a citation example! see references.bib as well
Happy writing \cite{alexandrov2004write}!